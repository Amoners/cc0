\documentclass[a4paper]{article}
% \usepackage[a4paper]{geometry}
\usepackage{fullpage}
\usepackage{times}
\usepackage{epsfig,endnotes}
\usepackage{algorithm}
\usepackage{algorithmic}
\usepackage{amssymb,amsmath}
\usepackage{color}
\usepackage{comment}
\usepackage{multirow}
\usepackage{graphicx,subfigure}
\usepackage{url}
\usepackage{alltt}
\usepackage{listings}
\usepackage[normalem]{ulem}
\lstset{basicstyle=\ttfamily\small, language=C, breaklines=true}


\begin{document}

\begin{flushleft}
\Large \textbf{C0}
\line(1, 0){450}
\end{flushleft}

\section{Introduction}
C0 is grammatically similar to the C language and will be immediately familiar to C, C++ and Java programmers. It is a procedure-oriented language with linguistic support for massive parallelism on a modern compute cluster. 

\subsection{Simple example}
Here we have a parallel version of vector addition it in C0.
\lstinputlisting[language=C]{examples/add.c}
The above program adds two vectors of length 10000 with 100 runners, each runner adds up 100 elements. A runner is a separate execution of code which is similar to threads.

\subsection{Program structure}
The four key concepts in C0 are programs, types, variables and functions. A program consists of one or more source file. Each source file defines some types or functions. The program must have a function named main with no parameter or return value. The main function is where the program starts.

\subsection{Keywords}
Note: The key words add yellow are not supported currently.

\begin{table}[htbp]
\centering
\caption{C0 key words}
\begin{tabular}{|c|c|c|c|}
\hline
abort & default & goto & static\\
\hline
auto & continue & if & struct\\
\hline
bool & double & int & switch\\
\hline
break & else & long & true\\
\hline
case & enum & return & unsigned\\
\hline
char & extern & runner & void\\
\hline
commit & commitd & false & volatile\\
\hline
const & float & signed & watching\\
\hline
while & for & do & register\\
\hline
in & sizeof & short & standalone\\
\hline
\end{tabular}
\label{table:key-words}
\end{table}

\section{Types}
There are several types in C0: simple types, struct types, union types, function types, void type, pointer types, array types, and array segments.

\subsection{Simple types}
Table 1 shows the simple types supported (Or would be supported) in C0.

\end{document}

